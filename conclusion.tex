\section{Conclusion}

In this report many different properties and behaviors of fluid flow through a diodic Tesla valve is 
studied. The flow is simulated for both laminar and turbulent behaviors and valve geometries and types are analyzed and interpreted. Lastly, an optimized straight segment length for the valvular conduit has 
been found and tested. It has been observed that this valve does show significantly diodic behavior without the need of an external energy source or any moving parts. \\ 

While the first task studied the velocity profiles at the inlet and outlets and comparing the with the analytical solution, Couette solution, it has also demonstrated the development of the boundary layer throughout the straight line segments at the inlets and outlets. \\


Although it took some time, the optimization of the straight line segment was quite similar to what Truong et al. reported in their study \cite{truong}. Furthermore, seeing that L value actually delivering the highest diodicity observed among all 18+ simulations made was surely reassuring. \\

Lastly, the same geometry was tested in three dimensions for the first time and turbulence models were used for this task. The results of the models were compared and the necessary plots like the pressure contours and streamlines are utilized to help with the discussion of the results.

