\subsection{Task 2: Investigation on the Effect of the Partition Straight Segment}
\label{sec:task2_desc}

This section revolves around observing and commenting on the effects of the partition straight segment, denoted $L$, on the pressure drop caused by the valve for different values of Reynolds number. Following this,  a correlation is to be found for the optimum straight segment length $L$,
normalized by the channel width $W$ as a function of the Reynolds number. \\

In their work, Truong et al \cite{truong}, have made a similar analysis and optimization on $\alpha$, of which we are using in this simulation to obtain optimal drop. However, in our case, we will be applying the same principles to $L$, which then will be tested thoroughly. 18 different simulations will be conducted for 3 different $L/W$ ratios, adjusting $L$ to be 200, 400 and 600mm while also changing the Reynolds Number for 3 different values and doing so in each direction for each configuration. The grid converged setup from Task-1 is used for these iterations. \\

$\alpha_{opt}$ will be used as given in the equation of Truong and Nam Trung \cite{truong} and pressure drops are measured by checking the pressure values at the center of the inlets and outlets for all cases, accounting for the switch of boundaries in each step.\\

Following our modeling of the system, simulations for Re = 724 in both flow directions with the corresponding $L_{opt}$, $\alpha_{opt}$, R and $\beta$ dimensions for the geometry will be conducted. Also, the same mesh size will be used as the tasks before. The new diodicity will be examined along with velocity plots, pressure plots and streamlines.